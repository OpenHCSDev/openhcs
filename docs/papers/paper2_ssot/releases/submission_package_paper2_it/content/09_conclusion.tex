\section{Conclusion}\label{sec:conclusion}
%==============================================================================

We have established a new capacity theorem extending zero-error source coding to interactive multi-location encoding systems. The key contributions are:

\textbf{1. Zero-Incoherence Capacity Theorem:} We define zero-incoherence capacity $C_0$ as the maximum encoding rate guaranteeing zero probability of location disagreement, and prove $C_0 = 1$ exactly (Theorem~\ref{thm:coherence-capacity}). The proof follows the achievability/converse structure of Shannon's channel capacity theorem.

\textbf{2. Side Information Bound:} We prove that resolution of $k$-way incoherence requires $\geq \log_2 k$ bits of side information (Theorem~\ref{thm:side-info}). This connects to Slepian-Wolf distributed source coding: provenance information acts as decoder side information.

\textbf{3. Multi-Terminal Interpretation:} We model encoding locations as terminals in a multi-terminal source coding problem. Derivation introduces perfect correlation (deterministic dependence), reducing effective rate. Only complete correlation (all terminals derived from one source) achieves zero incoherence.

\textbf{4. Rate-Complexity Tradeoffs:} We establish tradeoffs analogous to rate-distortion: $O(1)$ modification complexity at capacity vs. $\Omega(n)$ above capacity. The gap is unbounded (Theorem~\ref{thm:unbounded-gap}).

\textbf{5. Encoder Realizability:} Achieving capacity requires two encoder properties: causal propagation (analogous to feedback) and provenance observability (analogous to decoder side information). Both necessary; together sufficient (Theorem~\ref{thm:ssot-iff}).

\textbf{Corollary instantiations.} The abstract theory instantiates across domains (programming languages, distributed databases, configuration systems). Sections~\ref{sec:evaluation} and~\ref{sec:empirical} provide illustrative corollaries; the core theorems are domain-independent.

\textbf{Implications:}

\begin{enumerate}
\item \textbf{For information theorists:} Zero-error capacity theory extends to interactive multi-location encoding. The setting (modifiable encodings, coherence constraints) is new; the achievability/converse structure and side information bounds connect to established IT.

\item \textbf{For coding theorists:} Derivation is the mechanism that introduces correlation, reducing effective encoding rate. The encoder realizability theorem characterizes what encoder properties enable capacity-achieving codes.

\item \textbf{For system designers:} The capacity theorem is a forcing result: if coherence is required, encoding rate must be $\leq 1$. Systems operating above capacity require external side information for resolution.
\end{enumerate}

\textbf{Limitations:}
\begin{itemize}
\tightlist
\item Results apply primarily to facts with modification constraints. Streaming data and high-frequency updates have different characteristics.
\item The complexity bounds are asymptotic. For small encoding systems (DOF $< 5$), the asymptotic gap is numerically small.
\item Computational realization examples are primarily from software systems. The theory is general, but database and configuration system case studies are limited to canonical examples.
\item Realizability requirements focus on computational systems. Physical and biological encoding systems require separate analysis.
\end{itemize}

\textbf{Future Work:}
\begin{itemize}
\tightlist
\item \textbf{Probabilistic coherence:} Extend to soft constraints where incoherence probability is bounded but non-zero, analogous to the transition from zero-error to vanishing-error capacity.
\item \textbf{Network encoding:} Study coherence capacity in networked encoding systems with communication constraints, connecting to network information theory.
\item \textbf{Rate-distortion extension:} Characterize the full rate-complexity function $R(M)$ trading encoding rate against modification complexity, analogous to rate-distortion $R(D)$.
\item \textbf{Interactive capacity:} Study capacity under multi-round modification protocols, connecting to interactive information theory and directed information.
\item \textbf{Partial correlation:} Characterize coherence guarantees when derivation introduces partial (non-deterministic) correlation, extending beyond the perfect-correlation case.
\end{itemize}

\subsection{Artifacts}\label{sec:data-availability}

The Lean 4 formalization is included as supplementary material~\cite{openhcsLeanProofs}. The OpenHCS case study and associated code references are provided for the worked instantiation (Section~\ref{sec:empirical}).
